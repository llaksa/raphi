\documentclass{report}
\usepackage{graphicx}
\usepackage[utf8]{inputenc}
\usepackage[spanish]{babel}

\title{DISEÑO DE UN SISTEMA INTELIGENTE PARA AGRICULTURA VERTICAL}
\author{Irvin Pereyra}
\date{\today}

\begin{document}

\maketitle
\includegraphics{emacs.logo}
\section{Dedicatoria}
\section{Agradecimiento}
\pagenumbering{roman}
\tableofcontents
\newpage
\pagenumbering{arabic}
\section{Resumen}
...
...
...
El presente trabajo tiene como propósito integrar tres partes: Primero, un
módulo web de Internet de las Cosas para monitorización y control; Segundo, un
prototipo de PFC(Personal food computer) y tercero, un módulo de machine
learning orientado a optimizar el procese de cultivo hidropónico mediante las
técnicas A, B, C, etc.
...
...
...
\section{Abstract}

\chapter{Introducción}
\section{El problema}
\subsection{Realidad problemática}

La falta de educación, la desorganización y el uso inadecuado de los recursos
naturales por parte de nuestros agricultores genera desequilibrios en el
ecosistema que afectan a la agricultura sostenible. Así también, la casi nula
transferencia tecnológica y la creciente aparición de nuevos minifundios limitan
la explotación de terrenos entorpeciendo así los procesos logísticos y en
general, la dinámica del mercado agrícola.

La agricultura debe proveer los alimentos en la cantidad y la calidad necesarias
para una vida sana; no obstante, el tema de la seguridad alimentaria implica no
sólo mayor producción y productividad sino también una clara conciencia en los
consumidores sobre como alimentarse mejor. La insuficiencia de alimentos en
cantidad y calidad asociado con malos hábitos alimenticios repercuten sobre la
calidad de vida del habitante peruano, es por ello que los trabajos en materia
de seguridad alimentaria tienen aún mucho por delante.(Problemas en la
agricultura Peruana, MINAGRI, 15 de abril de 2018,  Disponible en:
http://minagri.gob.pe/portal/?id=190&start=3)

Desde un enfoque más global, es necesario darse cuenta que el mundo actual
necesita dejar de lado muchos sistemas socioeconómicos que van mostrando signos
de obsolencia. En torno a la agricultura, a parte de los procesos propios de
cultivo respecto de algún vegetal; existen muchos más factores a considerar para
garantizar el bienestar de los productores primarios y establecer una
agricultura sostenible a largo plazo. Por ejemplo: los más beneficiados con la
comercialización de los productos de primera necesidad, no son necesariamente
los agricultores, y sus ganacias siempre son afectadas por los costos elevados
del transporte. Y a pesar de que se planteen disintas reformas que puedan
aliviar un poco esta situación, la sobrepoblación y la escazes de recursos
naturales en los próximos años como el agua o el petróleo, terminará por obligar
a las sociedades a plantearse soluciones que la tecnología actual ya permite
hacer realidad. Además, puesto que dichas soluciones demuestran ser incluso más
eficientes que nuestro sistemas agrícolas tradicionales, es oportuno tratar de
implementarlas desde ahora.
\subsection{Antecedentes del problema}

`` Muchos estudios de producción hidropónica, aeropónica y acuaponia a escala
comercial mostraron el papel potencial positivo de esas nuevas tecnologías en la
seguridad alimentaria sostenible. Esos sistemas agrícolas podrían ser una
alternativa sostenible para proporcionar diferentes tipos de productos que
requieren menos agua, menos fertilizantes y menos espacio, lo que aumentará el
rendimiento por unidad de área'' (Ali AlShrouf, 2017, p.1).

``Nuestros resultados demuestran claramente que los sistemas de cultivo
verticales (VFS) presentan una alternativa atractiva a los sistemas de
crecimiento hidropónico horizontal y sugieren que se podrían lograr mayores
aumentos en el rendimiento mediante la incorporación de iluminación artificial
en el VFS.'' (Dionysios Touliatos, Ian C. Dodd & Martin McAinsh, 2016, p. 1)

\subsection{Formulación del problema}

¿Qué condiciones ambientales son determinantes en la producción de
cultivos verticales?

\subsection{Justificación del estudio}
BUSCAR PUBLICACIONES CIENTÍFICAS Y CITAR AQUÍ LO MÁS QUE SE PUEDAAAAAAAAAA!!!
\begin{enumerate}
\item[a] Relevancia Tecnológica
\item[b] Relevancia Institucional
\item[c] Relevancia Social
\item[d] Relevancia Económica
\item[e] Relevancia Ambiental
\end{enumerate}
\subsection{Limitaciones del problema}
asdfr
\section{Objetivos}
asdfr
\subsection{Objetivo general}
ar
\subsection{Objetivos específicos}
asdfr
\chapter{Marco referencial}
asdfr
\section{Marco teórico}
asdfr
\section{Marco conceptual}
asdfr
\chapter{Metodología}
asdfr
\section{Hipótesis}
asdfr
\section{Variables}
asdfr
\subsection{Variable dependiente}
asdfr
\subsection{Variable independiente}
asdfr
\section{Diseño de ejecución}
asdfr
\subsection{Objeto de estudio}
asdfr
\subsection{Métodos}
asdfr
\subsection{Población y muestra}
asdfr
\subsection{Técnicas e Instrumentos, fuentes e informantes}
asdfr
\subsection{Forma de análisis e Interpretación de resultados}
asdfr
\subsubsection{Análisis de contrastación}
asdfr
\subsubsection{Indicadores}
asdfr
\chapter{Resultados}
asdfr
\chapter{Análisis de los resultados}
asdfr
\chapter{Conclusiones y recomendaciones}
asdfr
\chapter{Referencias bibliográficas}
asdfr
\chapter{Anexos}
asdfr

\end{document}