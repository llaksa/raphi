\documentclass{report}
\usepackage{graphicx}
\usepackage[utf8]{inputenc}
\usepackage[spanish]{babel}

\title{DISEÑO DE UN SISTEMA INTELIGENTE PARA AGRICULTURA VERTICAL}
\author{Irvin Pereyra}
\date{\today}

\begin{document}

\maketitle
\includegraphics{emacs.logo}
\section{Dedicatoria}
\section{Agradecimiento}
\pagenumbering{roman}
\tableofcontents
\newpage
\pagenumbering{arabic}
\section{Resumen}
El presente trabajo tiene como propósito integrar tres partes: Primero, un
módulo web de Internet de las Cosas para monitorización y control; Segundo, un prototipo de PFC(Personal food
computer) y tercero, un módulo de machine learning orientado a optimizar el procese
de cultivo mediante las técnicas A, B, C, etc.
...
...
...
\section{Abstract}

\chapter{Introducción}

\section{El problema}

\subsection{Realidad problemática}


The Free Dictionary. Farlex, ©2003-2016 [consulta: 26 febrero 2016]. Disponible en: http://www.thefreedictionary.com/

PROBLEMAS TIPO DE LA AGRICULTURA PERUANA

a. Conservación del Medio Ambiente Erosión y Salinización

El Perú es uno de los doce países considerados como megadiversos y se estima que posee entre 60 y 70% de la diversidad biológica. Esta ventajosa situación se ha visto amenazada con un inadecuado manejo de recursos existentes llevándolo a niveles críticos de deterioro de ciertas zonas del país generando problemas de desertificación, deforestación, salinización, pérdida de tierras agrícolas, toxicidad de la vegetación, agotamiento de las fuentes de agua, degradación de ecosistemas y desaparición de especies silvestres.

La situación de pobreza de la mayor parte de campesinos y pequeños productores agropecuarios se explican en parte por la utilización inadecuada y degradación de la base productiva de los recursos naturales debido a la aplicación de sistemas productivos que generan desequilibrios negativos entre el proceso de extracción y regeneración de los recursos naturales.

Promover acciones para el manejo y uso productivo de los recursos naturales renovables, agua, suelo y cobertura vegetal mediante obras de conservación de suelos, reforestación, transferencia tecnológica mejorada e infraestructura rural en la perspectiva de lograr una agricultura sostenible

b. Minifundio

La agricultura peruana constituye una economía de parceleros en la cual el 85% de los agricultores tiene parcelas con menos de 10 hectáreas predominando las unidades productivas con un área entre 3 y 10 hectáreas (33%) (. Existen 5.7 millones de predios rurales de los cuales figuran inscritos en registro públicos solamente un tercio (1.9 millones). Lo más grave es que el minifundio sigue creciendo.

El fraccionamiento de las parcelas en pequeños minifundios y su gran dispersión representan un límite a la eficiencia productiva al tiempo que eleva los costos del transporte.

La tierra es el principal activo que posee el agricultor por lo que sus derechos
de propiedad deben estar claramente definidos a fin de que ese reconocimiento
legal les proporciones respaldo a la producción.

c. Preciosciences y mercados

La actividad agraria se caracteriza por el desorden en la producción y la disminución de su rentabilidad y competitividad. Asimismo, los procesos de post cosecha y de mercadeo están sumamente desordenados por la falta de una infraestructura vial adecuada y la ausencia de un sistema de mercados mayoristas, lo cual incide en los altos costos de comercialización que afectan a los productores agrarios.
Una característica del mercadeo interno agrícola es la multiplicidad de intermediarios. Geoffrey Cannock y Alberto Gonzales-Zuñiga en su Libro “Economía Agraria” mencionan al menos siete eslabones de intermediación:
1. El productor.-Quien normalmente mercadea un escaso volumen de producción, no están organizados para mercadear sus producto puede enfrentarse a situaciones de monopsonio y oligopsonio.
2. El acopiador, es el comerciante local.
3. El transportista que actúa como rescatista.
4. El mayorista, generalmente está especializado por producto.
5. El distribuidor que reparte el producto a través de sus canales de minoristas.
6. Los minoristas, quienes están generalmente muy dispersos y tienen poca capacidad de negociación frente a los distribuidores y mayoristas, operan en los mercados públicos, de barrio y en las calles.
7. El consumidor
Por norma general, opinan ambos autores, “puede afirmarse que el sector de mercadeo interno enfrenta altos costos, problemas de escala, altas mermas, carencia de infraestructura, todos estos factores lo tornan ineficiente.”

Un sistema de comercialización eficiente representa una de las claves para favorecer una correcta formación de precios en función de las fuerzas del mercado.

d. Asistencia Técnica

Las tareas en el campo de la asistencia técnica son múltiples, consolidar el
crecimiento agrario exigirá el desarrollo de factores productivos y el impulso
de la innovación tecnológica, es por ello que una de las tareas es atender las
necesidades urgentes de los productores en materias de Innovación tecnológica y
gestión empresarial.

e. Crédito Agrario

El tema del crédito representa uno de los cuellos de botella del sector, es por ello que el anuncio de la creación de un Banco es esperado con mucha Interés por la mayoría de agentes económicos.
En 1992 el banco Agrario tenía como clientes a 230 mil empresarios agrarios que se vinculaban mediante operaciones directas, de ellos 20,000 productores eran de tipo A1, con 10,20,30 ó 40 años de tradición empresarial.
La banca comercial es la principal fuente de financiamiento del sector y el 86% de sus colocaciones están en Lima. La mitad de dichas colocaciones son de corto plazo lo que dificulta la capitalización de sectores como la agricultura. También participan en el financiamiento a agricultores los comerciantes, los habilitadores y transportistas en menor escala.
El tema de financiamiento agrario deberá enfrentar numerosos retos en la búsqueda por una agricultura en expansión sostenida en el tiempo y sustentable desde el punto de vista ambiental. Entre otros podemos mencionar:

-Acceso al crédito.
-El costo del crédito, el cual suele ser superior al resto de la economía debido al mayor riesgo.
-Su uso racional desde el punto de vista económico.
-Incorporación al mercado financiero de millares de agentes productivos sin crédito.
-La recuperabilidad de los créditos tema fundamental pues tiene que ver con la viabilidad de largo plazo del sistema de crédito y con la rentabilidad de la actividad agraria. Un sistema de crédito agrario basado en el subsidio, vía menores tasas de interés y con altos grados de incobrabilidad no es sostenible en el tiempo; tienen elevados costos sociales y por lo general no cumple los objetivos trazados.

f. Organizaciones

El fortalecimiento de las organizaciones de productores y de otras organizaciones que contribuyan al desarrollo del agro representa una tarea impostergable; en un sector tan complejo la capacidad organizativa y de cooperación Inter e Intra sectorial representan importantes instrumentos de competitividad. “Muchas instituciones agrarias son vistas ahora como respuestas coherentes a la falta de desarrollo adecuado de los mecanismos del mercado, tales como el crédito, seguro agrario y mercados a futuro en un contexto caracterizado por altos riesgos, asimetrías de información y riesgo moral” ( Cannock, Geoffrey; Gonzales-Zúñiga, A. Economía Agraria).

Apoyar a las organizaciones de los empresarios agrarios permitirá afianzar el
planeamiento de las cadenas productivas que representa una parte sustantiva de
la actividad agraria del país.

g. Seguridad Alimentaria

La agricultura debe proveer los alimentos en la cantidad y la calidad necesarias para una vida sana; no obstante, el tema de la seguridad alimentaria implica no sólo mayor producción y productividad sino también una clara conciencia en los consumidores sobre como alimentarse mejor. La insuficiencia de alimentos en cantidad y calidad asociado con malos hábitos alimenticios repercuten sobre la calidad de vida del habitante peruano, es por ello que los trabajos en materia de seguridad alimentaria tienen aún mucho por delante.

h. El Empleo

La agricultura emplea al 26% de la PEA Nacional y al 65.5% de la PEA del área rural. En contraste con su capacidad de generar empleo, es uno de los sectores con menor productividad de mano de obra debido al bajo nivel educativo de la fuerza laboral en el ámbito rural

i. Sanidad

Países como Chile muestran lo importante que resulta para el desarrollo del
sector agrario el contar con una buen sistema de sanidad animal y vegetal,
especialmente si existe la proyección hacia la exportación. Cautelar la
seguridad sanitaria y fitosanitaria posibilitando el desarrollo de cosechas y
crianzas sanas; controlar y erradicar las plagas y enfermedades representan
acciones con una enorme incidencia socieconómica en la actividad agraria. Un
sistema sanitario eficaz es al mismo tiempo funcional al desarrollo exportador.
Una de las principales limitaciones para el acceso a los mercados externos se
relaciona con problemas sanitarios como es el caso de la mosca de la fruta.


\subsection{Antecedentes del problema}
adfr
\subsection{Formulación del problema}

¿Qué efecto tiene la implementación de tecnologías actuales sobre los procesos agrícolas urbano-doméstico tradicionales?

\subsection{Justificación del estudio}
BUSCAR PUBLICACIONES CIENTÍFICAS Y CITAR AQUÍ LO MÁS QUE SE PUEDAAAAAAAAAA!!!
\begin{enumerate}
\item[a] Relevancia Tecnológica
\item[b] Relevancia Institucional
\item[c] Relevancia Social
\item[d] Relevancia Económica
\item[e] Relevancia Ambiental
\end{enumerate}
\subsection{Limitaciones del problema}
asdfr
\section{Objetivos}
asdfr
\subsection{Objetivo general}
ar
\subsection{Objetivos específicos}
asdfr
\chapter{Marco referencial}
asdfr
\section{Marco teórico}
asdfr
\section{Marco conceptual}
asdfr
\chapter{Metodología}
asdfr
\section{Hipótesis}
asdfr
\section{Variables}
asdfr
\subsection{Variable dependiente}
asdfr
\subsection{Variable independiente}
asdfr
\section{Diseño de ejecución}
asdfr
\subsection{Objeto de estudio}
asdfr
\subsection{Métodos}
asdfr
\subsection{Población y muestra}
asdfr
\subsection{Técnicas e Instrumentos, fuentes e informantes}
asdfr
\subsection{Forma de análisis e Interpretación de resultados}
asdfr
\subsubsection{Análisis de contrastación}
asdfr
\subsubsection{Indicadores}
asdfr
\chapter{Resultados}
asdfr
\chapter{Análisis de los resultados}
asdfr
\chapter{Conclusiones y recomendaciones}
asdfr
\chapter{Referencias bibliográficas}
asdfr
\chapter{Anexos}
asdfr

\end{document}