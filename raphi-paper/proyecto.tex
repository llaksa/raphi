\documentclass{report}
\usepackage{graphicx}
\usepackage[utf8]{inputenc}
\usepackage[spanish]{babel}

\date{\today}

\begin{document}

\pagenumbering{roman}
\tableofcontents
\pagenumbering{arabic}

\chapter{GENERALIDADES}
\section{Título}
Diseño de un módulo de software para optimizar parámetros climáticos en un
cultivo doméstico de ambiente controlado
\section{Autor}
Irvin Jair Pereyra Gonzáles
\section{Asesor}
Mg. Javier León Lescano
\section{Tipo de investigación}
\subsection{De acuerdo al fin que se persigue}
  Investigación aplicada.
\subsection{De acuerdo a la técnica de contrastación}
  Investigación experimental.
\subsection{De acuerdo al régimen de investigación}
  Investigación orientada.
\section{Localidad}
Ciudad de Trujillo
\section{Duración del proyecto}
Fecha de inicio: Junio del 2018
Fecha de término: Agosto del 2018
\chapter{PLAN DE INVESTIGACIÓN}
\section{Realidad problemática}
La falta de educación, la desorganización y el uso inadecuado de los recursos
naturales por parte de muchos agricultores genera desequilibrios en el
ecosistema que afectan a la agricultura sostenible. Así también, la casi nula
transferencia tecnológica y la continua aparición de nuevos minifundios limitan
la explotación de terrenos; entorpeciendo los procesos logísticos y en general,
la dinámica del mercado agrícola.

La agricultura debe proveer los alimentos en la cantidad y la calidad necesarias
para una vida sana; no obstante, el tema de la seguridad alimentaria implica no
sólo mayor producción y productividad sino también una clara conciencia en los
consumidores sobre como alimentarse mejor. La insuficiencia de alimentos en
cantidad y calidad asociado con malos hábitos alimenticios repercuten sobre la
calidad de vida del habitante peruano, es por ello que los trabajos en materia
de seguridad alimentaria tienen aún mucho por delante.(Problemas en la
agricultura Peruana, MINAGRI, 15 de abril de 2018,  Disponible en:
http://minagri.gob.pe/portal/?id=190&start=3)

Desde un enfoque más global, es necesario darse cuenta que el mundo actual
necesita dejar de lado muchos sistemas socioeconómicos que van mostrando signos
de obsolescencia. En torno a la agricultura, a parte de los procesos propios de
cultivo respecto de algún vegetal; existen muchos más factores a considerar para
garantizar el bienestar de los productores primarios y establecer una
agricultura sostenible a largo plazo. Por ejemplo, los más beneficiados con la
comercialización de los productos de primera necesidad, no son necesariamente
los agricultores, y sus ganancias siemprhttp://rpp.pe/?ref=rppe son afectadas por los costos elevados
del transporte. Y a pesar de que se planteen distintas reformas que puedan
aliviar un poco esta situación, la sobrepoblación y la escasez de recursos
naturales en los próximos años, como el agua o el petróleo, terminará por
obligar a las sociedades a plantearse soluciones que la tecnología actual ya
permite hacer realidad. Además, puesto que dichas soluciones demuestran ser
incluso más eficientes que nuestro sistemas agrícolas tradicionales, es oportuno
implementarlas desde ahora para que así se pueda ir mejorando y adecuando a
nuestra realidad.
\section{Formulación del problema}
¿Cómo determinar mejores condiciones climáticas para un cultivo en sistemas
agrícolas de ambiente controlado?
\section{Justificación}
\subsection{Relevancia Tecnológica}
  
  En centros de investigación, desarrollo e innovación alrededor del
  mundo se están implementando nuevas alternativas como la presentada en este
  trabajo, que cambian y demuestran resultados prometedores respecto del sistema
  con el que se ha venido manejando la agricultura tradicional. Por lo tanto, es
  conveniente aprovechar la gran variedad de opciones tecnológicas disponibles
  en el mercado para comenzar a explorar los beneficios de su aplicación en
  nuestra realidad.
  
\subsection{Relevancia Institucional}
  
  El presente trabajo tiene la capacidad de atraer equipos de investigación
  multidisciplinarios. Por ejemplo, que conlleven la participación de
  interesados en los campos de la física, matemática, ciencias de la
  computación, ingeniería, gestión, economía, etc.
  
  \subsection{Relevancia Social}
  
  Con los nuevos métodos agrícolas se pretende también de que cada individuo
  viva participando activamente en un entorno donde las actividades agrícolas
  sean autosostenibles. Lo que implica, a su vez, una elevación de la cultura
  alimentaria.
  
\subsection{Relevancia Económica}
  
  La agricultura vertical también implica desligar los procesos logísticos y de
  transporte que actualmente son imprescindibles para que los agricultores
  puedan vender sus productos, y que actualmente es común que la variación del
  costo del petróleo influya negativamente en sus ganancias.
  
\subsection{Relevancia Ambiental}
  
  Kurt Benke y Bruce Tomkins (2017) afirman que Las megatendencias mundiales de
  la disminución del suministro de agua, el aumento de la población, la
  urbanización y el constante cambio climático han contribuido a la disminución
  global de las existencias de tierras cultivables por persona. En estas
  circunstancias, es probable que la sostenibilidad del modelo agrícola
  tradicional basado en grandes granjas rurales se vea amenazada en las próximas
  décadas. Un enfoque para abordar este problema desafiante es la agricultura
  vertical, que se basa en la agricultura de ambiente controlado y diseños de
  invernadero adecuados para entornos urbanos (p. 12). La agricultura vertical
  se ha demostrado a escala piloto y también a nivel de producción y tiene
  ventajas potenciales sobre la agricultura rural, incluido el uso de la
  hidroponía, que desafía la necesidad de una agricultura basada en el suelo
  para una variedad de cultivos. Los beneficios potenciales de la agricultura
  vertical incluyen un modelo sostenible de producción de alimentos con
  producción de cultivos durante todo el año, mayores rendimientos en un orden
  de magnitud y ausencia de sequías, inundaciones y plagas (p. 13).
\section{Antecedentes}
Muhammad Ikhwan y Norashikin M. Thamrin (2018) presentan un proyecto cuyo
objetivo principal es construir un sistema para controlar la humedad del suelo y
controlar el contenido de agua a través del navegador web en la computadora
portátil, el teléfono móvil y otros dispositivos portátiles y compactos (p. 1).

Deepak Vasisht, Zerina Kapetanovic, Jongho Won, Xinxin Jin, Ranveer Chandra,
Ashish Kapoor, Sudipta N. Sinhaand Madhusudhan Sudarshan, Sean Stratman (2017)
afirman que las técnicas basadas en datos ayudan a impulsar la productividad
agrícola al aumentar los rendimientos, reducir las pérdidas y reducir los costos
de los insumos. Sin embargo, estas técnicas han visto una adopción escasa debido
a los altos costos de recopilación manual de datos y soluciones de conectividad
limitadas (p. 1).

Yap Shien Chin y Lukman Audah (2017) afirman que la agricultura vertical es
difícil de practicar porque los cambios menores en el entorno dejarían un gran
impacto en la productividad y la calidad de la actividad agrícola. Por lo que,
presentan un estudio con el objetivo de proporcionar un sistema de monitoreo
agrícola vertical para ayudar a mantener el seguimiento de las condiciones
físicas de los cultivos (p. 1).

Kurt Benke y Bruce Tomkins (2017) afirman que existe la necesidad de aumentar
los fondos para la investigación en genética vegetal para optimizar el
rendimiento, ampliando la gama de tipos de cultivos y ajustando para obtener una
respuesta óptima a variables controladas como la longitud de onda de la
iluminación LED, la temperatura, la humedad y los niveles de CO2 (p. 14).

Los ``resultados demuestran claramente que los sistemas de cultivo
verticales (VFS) presentan una alternativa atractiva a los sistemas de
crecimiento hidropónico horizontal y sugieren que se podrían lograr mayores
aumentos en el rendimiento mediante la incorporación de iluminación artificial
en el VFS.'' (Dionysios Touliatos, Ian C. Dodd y Martin McAinsh, 2016, p. 1).

Malek Al-Chalabi (2015) afirma que los hallazgos indican que la agricultura
vertical es un concepto que está en su infancia técnica pero que promete para
las ciudades futuras. La investigación adicional puede ayudar a continuar con
esta idea. Esto incluye desarrollar diseños multifuncionales con aportes de
ingenieros, arquitectos y proveedores de tecnología agrícola vertical
simultáneamente para ayudar a diseñar estructuras futuras que puedan adaptarse a
las necesidades del siglo XXI, desarrollando programas piloto donde se puedan
recopilar y analizar datos en tiempo real para examinar dónde existen
oportunidades y barreras en comparación con los productos convencionales, el
desarrollo de un modelo de energía más grande que pueda tener más factores en
cuenta (ventilación, desperdicio, etc.) y la realización de un estudio
tecnoeconómico que incorpora los costos de construcción y mantenimiento.
La agricultura vertical tiene potencial en las circunstancias correctas. En esos
casos y con un poco más de investigación, el cielo es el límite. (p. 4).
\section{Objetivo}
\subsection{General}
Determinar los métodos que permitan optimizar la producción de cultivos en
sistemas agrícolas de ambiente controlado.
\subsection{Específicos}
\begin{enumerate}
\item[-] Determinar qué componentes tecnológicos permiten medir parámetros
  ambientales importantes para la producción de un cultivo.
\item[-] Determinar qué componentes tecnológicos permiten manipular parámetros
  ambientales importantes para la producción de un cultivo.
\item[-] Describir los valores actuales de los parámetros ambientales de un
  cultivo.
\item[-] Indicar los valores deseados para los parámetros ambientales de un
  cultivo.
\item[-] Relacionar los valores deseados con los valores actuales.
\item[-] Analizar los valores medidos para hallar datos que optimicen el cultivo.
\item[-] Relacionar los valores optimizados con los valores deseados.
\end{enumerate}
\section{Marco teórico}
\section{Marco conceptual}

\chapter{METODOLOGÍA}
\section{Tipo de estudio}
Aplicado y exploratorio
\section{Diseño de investigación}
Experimental pura
\section{Hipótesis}
Mediante la monitorización, control y registro de ciertos parámetros ambientales
de un cultivo se podrá utilizar técnicas de algoritmos genéticos y visión
artificial para optimizar la producción del mismo en sistemas agrícolas de
ambiente controlado.
\section{Identificación de variables}
VARIABLES DEPENDIENTES
Temperatura de aire
Volumen de agua
Intensidad de Luz
Periodicidad de injección de aire fresco
Periodicidad de injección de agua fresca
Periodicidad de circulación de aire
Periodicidad de circulación de agua

VARIABLES INDEPENDIENTES
Color y tamaño de la planta cultivada
Temperatura del agua
Monóxido de oxígeno
\subsection{Operacionalización de variables}
Temperatura de aire
Temperatura del aire al rededor de la planta dentro del prototipo de PFC
Medible a través de un sensor de temperatura y alterable por medio de un sistema
de control
Temperatura
15 °C - 25 °C

Volumen de agua
Volumen del agua sobre la cual está la planta dentro del prototipo de PFC
Medible indirectamente a través de un sensor de distancia y alterable por medio
de un sistema de control
Longitud
0 cm - 20 cm

Intensidad de Luz
Intensidad de luz dentro del prototipo de PFC
Intensidad luminosa
0 lx - 1000 lx

Periodicidad de injección de aire fresco
Frecuencia con la que el prototipo de PFC es administrada con aire fresco
Alterable a través de la programación de intervalos de tiempo en un sistema de control
Tiempo
c / 2 h  - c / 4 h

Periodicidad de injección de agua fresca
Frecuencia con la que el prototipo de PFC es administrada con agua fresca
Alterable a través de la programación de intervalos de tiempo en un sistema de
control
Tiempo
c / 2 h  - c / 4 h

Periodicidad de circulación de aire
Frecuencia con la que se hace circular el aire en el prototipo de PFC
Alterable a través de la programación de intervalos de tiempo en un sistema de
control
Tiempo
c / 2 h  - c / 4 h

Periodicidad de circulación de agua
Frecuencia con la que se hace circular el agua en el prototipo de PFC
Alterable a través de la programación de intervalos de tiempo en un sistema de
control
Tiempo
c / 2 h  - c / 4 h

Color del cultivo
Color de las plantas cultivadas en el prototipo de PFC
Medible a través de la aplicación de algoritmos de visión artificial sobre una
imagen digital diaria de las plantas
Color
rgb(0,0,0) - rgb(255,255,255)

Área del cultivo
Área de las plantas cultivadas en el prototipo de PFC
Medible indirectamente a través de la aplicación de algoritmos de visión artificial sobre una
imagen digital diaria de las plantas
Longitud^2
0 m^² - 0.5 m^²

Temperatura del agua
Temperatura del agua sobre la que la planta está dentro del prototipo de PFC
Medible a través de un sensor de temperatura
Temperatura
15 °C - 25 °C

Nivel de Monóxido de oxígeno
Concentración de monóxido de carbono en el prototipo de PFC
Medible a través de un sensor de la concentración de CO
Concentración
20 ppm - 2000 ppm
\section{Población, muestra y muestreo}
\subsection{Población}
Sistemas robóticos para agricultura personal
\subsection{Muestra}
Computadoras personales alimentarias
\subsection{Unidad de análisis}
Prototipo propio de una computadora alimentaria personal
\section{Criterios de selección}
\subsection{Criterios de inclusión}
Sistemas customizables, adaptativos (que permitan cambiar el ambiente al rededor de
la planta), de bajo costo e información libre (open-source software/hardware).
\subsection{Criterios de exclusión}
Sistemas no custommizables, no adaptativos, con alto costo o con
software/hardware privativo.
\section{Método de investigación}
Método heurístico y experimental.
\section{Técnicas de recolección de datos}
Los datos serán obtenidos mediante bibliografía especializada.
\section{Procedimientos de recolección de datos}
Elegir las fuentes de datos de libros y repositorios digitales de artículos
científicos.

Recopilar información sobre sistemas robóticos agrícolas, sistemas de cultivo de
ambiente controlado y sistemas hidropónicos.

Extraer y organizar las características ambientales adecuadas que deben
conservarse para fines agrícolas.
\section{Métodos de análisis de datos}
Se analizarán los datos mediante la cuantificación de determinados parámetros
ambientales y características del cultivo.
\chapter{ASPECTOS ADMINISTRATIVOS}
\section{Recursos y presupuesto}
\subsection{Recursos}
\begin{enumerate}
\item[-] Humanos:
  Asesor de tesis.
  Bachiller de ingeniería mecatrónica.
\item[-] Materiales y Equipos:
  
\end{enumerate}
\subsection{Presupuesto}
\section{Financiamiento}
Autofinanciamiento.
\section{Cronograma de ejecución}

\chapter{REFERENCIAS BIBLIOGRÁFICAS}
Muhammad Ikhwan y Norashikin M. Thamrin (2018). 2017 International Conference on
Electrical, Electronics and System Engineering (ICEESE). Kanazawa: IEEE.

Kurt Benke and Bruce Tomkins (2017). Future food-production systems: vertical
farming and controlled-environment agriculture. United Kingdom: Informa UK
Limited.

Yap Shien Chin y Lukman Audah (2017). Vertical farming monitoring system using
the internet of things (IoT), Malaysia: Universiti Tun Hussein Onn Malaysia.

Deepak Vasisht, Zerina Kapetanovic, Jongho Won, Xinxin Jin, Ranveer Chandra,
Ashish Kapoor, Sudipta N. Sinhaand Madhusudhan Sudarshan, Sean Stratman (2017).
Proceedings of the 14th USENIX Symposium on Networked Systems Design and
Implementation (NSDI ’17).

Dionysios Touliatos, Ian C. Dodd y Martin McAinsh (2016). Vertical farming
increases lettuce yield per unit area compared to conventional horizontal
hydroponics, UK: he Lancaster Environment Centre, Lancaster University,
Lancaste.

Malek Al-Chalabi (2015). Vertical farming: Skyscraper sustainability?, United
Kingdom: Elsevier.
\chapter{ANEXOS}
\end{document}