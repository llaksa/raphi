\documentclass{report}
\usepackage{graphicx}
\usepackage[utf8]{inputenc}
\usepackage[spanish]{babel}

\title{SOME NOTES}
\author{Me}
\date{\today}

\begin{document}

\section{About Hydroponics}

\chapter{Cultivo en hidroponía (José Beltrano y Daniel O. Gimenez )}

- El desarrollo actual de la técnica de los cultivos hidropónicos, está basada
en la utilización de mínimo espacio, mínimo consumo de agua y máxima producción
y calidad.

Ventajas de los cultivos hidropónicos:

Cultivos libres de parásitos, bacterias, hongos y contaminación.
Reducción de costos de producción.
Independencia de los fenómenos meteorológicos.
Permite producir cosechas en contra estación
Menos espacio y capital para una mayor producción.
Ahorro de agua, que se puede reciclar.
Ahorro de fertilizantes e insecticidas.
Se evita la maquinaria agrícola (tractores, rastras, etcétera).
Limpieza e higiene en el manejo del cultivo.
Mayor precocidad de los cultivos.
Alto porcentaje de automatización.
Mejor y mayor calidad del producto.
Altos rendimientos por unidad de superficie
Aceleramiento en el proceso de cultivo
Posibilidad de cosechar repetidamente la misma especie de planta al año
Ahorro en el consumo del agua
Productos libres de químicos no nutrientes.

Desventajas:

El costo inicial.

Control de Malezas, plagas y enfermedades:

La hidroponía elimina la posibilidad del suelo infestado con plagas. Dado que no utiliza
el suelo, no hay lugar para que las malezas compitan con el cultivo. Desafortunadamente,
puede ser rápida la propagación de enfermedades de plantas en los sistemas hidropónicos.
Dado que las plantas o el cultivo está conectado por el sistema de suministro de agua y
nutrientes, una planta enferma introducida en el sistema puede propagar rápidamente su
problema a todas las demás.